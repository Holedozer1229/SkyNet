\documentclass{article}
\usepackage{amsmath}
\usepackage{amssymb}
\usepackage{geometry}
\geometry{a4paper, margin=1in}

\title{StarLord2 × SKYNT LaunchNFT Ecosystem\\Mathematical Formulas}
\author{StarLord2 Team}
\date{\today}

\begin{document}

\maketitle

\section{Introduction}

This document contains all symbolic and algebraic formulas used in the StarLord2 × SKYNT LaunchNFT ecosystem.

\section{Φ (Phi) Computation}

\subsection{Total Φ Value}

The total Φ value is computed as the weighted average of individual φ parameters:

\begin{equation}
\Phi_{\text{total}} = \frac{\sum_{i=1}^{N} w_i \cdot \varphi_i}{N}
\end{equation}

where:
\begin{itemize}
    \item $w_i$ is the weight of parameter $i$
    \item $\varphi_i$ is the value of parameter $i$
    \item $N$ is the total number of parameters
\end{itemize}

\subsection{Delta-S Total}

The total shift in the system is computed as:

\begin{equation}
\Delta S_{\text{total}} = \Delta S_{\text{geom}} + \Delta S_{\text{protocol}}
\end{equation}

where:

\begin{equation}
\Delta S_{\text{geom}} = \sum_{i=1}^{N} \sqrt{w_i \cdot \varphi_i}
\end{equation}

\begin{equation}
\Delta S_{\text{protocol}} = \frac{1}{N}\sum_{i=1}^{N} e_i
\end{equation}

and $e_i$ represents the eigenvector component for parameter $i$.

\section{M-Shift Optimization}

\subsection{Eigenvector Particle Flow}

The particle positions in 3D space are computed using the M-shift optimization with eigenvector influence:

\begin{align}
x &= \left[ (\text{rand} - 0.5) \cdot 200 + \text{val} \cdot 2 \right] \cdot e_i \\
y &= \left[ (\text{rand} - 0.5) \cdot 200 + \text{val} \cdot 1.5 \right] \cdot e_i \\
z &= \left[ (\text{rand} - 0.5) \cdot 200 + \text{val} \cdot 3 \right] \cdot e_i
\end{align}

where:
\begin{itemize}
    \item $\text{rand} \in [0, 1]$ is a uniform random variable
    \item $\text{val}$ is the normalized parameter value
    \item $e_i$ is the eigenvector for parameter $i$
\end{itemize}

\section{NFT Rarity System}

\subsection{Rarity Score}

The rarity score for NFT $i$ is computed as:

\begin{equation}
R_i = f(\text{supply}, \text{demand}, \Phi_i) = \frac{\text{demand} \cdot \Phi_i}{\text{supply} + 1}
\end{equation}

where:
\begin{itemize}
    \item $\text{demand}$ is the market demand at mint time
    \item $\Phi_i$ is the Φ value when the NFT was minted
    \item $\text{supply}$ is the total NFT supply at mint time
\end{itemize}

\subsection{Rarity Tiers}

NFTs are classified into rarity tiers based on their score $R_i$:

\begin{equation}
\text{Tier}(R_i) = 
\begin{cases}
\text{Legendary} & \text{if } R_i \geq 99 \\
\text{Epic} & \text{if } 90 \leq R_i < 99 \\
\text{Rare} & \text{if } 75 \leq R_i < 90 \\
\text{Uncommon} & \text{if } 50 \leq R_i < 75 \\
\text{Common} & \text{if } R_i < 50
\end{cases}
\end{equation}

\section{Staking Rewards}

\subsection{Reward Calculation}

The staking rewards for a user are computed as:

\begin{equation}
\text{rewards} = \frac{\text{amount} \cdot r \cdot t}{10000 \cdot 365 \text{ days}}
\end{equation}

where:
\begin{itemize}
    \item $\text{amount}$ is the staked amount
    \item $r$ is the reward rate in basis points
    \item $t$ is the time staked in seconds
\end{itemize}

\section{Omega Infinite Yield}

\subsection{Omega Multiplier}

The Omega Infinite multiplier increases yield over time:

\begin{equation}
\Omega(t) = 1 + \frac{t_{\text{days}}}{365} \cdot 0.5
\end{equation}

where $t_{\text{days}}$ is the number of days staked.

\subsection{Total Yield}

The total yield with Omega multiplier is:

\begin{equation}
Y_{\text{total}} = (Y_{\text{staking}} + Y_{\text{attack}} + Y_{\text{NFT}}) \cdot \Omega(t)
\end{equation}

\section{Attack System}

\subsection{Attack Damage}

The damage dealt by an attack is:

\begin{equation}
D = \frac{D_{\text{base}} \cdot \Phi_{\text{current}}}{1000}
\end{equation}

where:
\begin{itemize}
    \item $D_{\text{base}}$ is the base damage for the attack tier
    \item $\Phi_{\text{current}}$ is the current Φ value
\end{itemize}

\subsection{Energy Regeneration}

Energy regenerates linearly over time:

\begin{equation}
E(t) = \min\left(E_0 + \frac{t \cdot r_E}{1 \text{ hour}}, E_{\max}\right)
\end{equation}

where:
\begin{itemize}
    \item $E_0$ is the initial energy
    \item $r_E$ is the regeneration rate (energy per hour)
    \item $E_{\max}$ is the maximum energy capacity
    \item $t$ is time elapsed in hours
\end{itemize}

\section{Raid Pass System}

\subsection{Staking Weight Multiplier}

Raid passes provide staking weight multipliers:

\begin{equation}
w_{\text{effective}} = w_{\text{base}} \cdot m_{\text{tier}}
\end{equation}

where:
\begin{itemize}
    \item $w_{\text{base}}$ is the base staking weight
    \item $m_{\text{tier}} \in \{1.0, 1.5, 2.0\}$ is the tier multiplier
\end{itemize}

\section{Cross-Chain Mining}

\subsection{Hashrate Distribution}

The hashrate share for chain $c$ is:

\begin{equation}
h_c = \frac{H_c}{\sum_{i=1}^{C} H_i}
\end{equation}

where $H_c$ is the hashrate for chain $c$ and $C$ is the total number of chains.

\subsection{Mining Efficiency}

The efficiency metric for a chain is:

\begin{equation}
\eta_c = \frac{H_c / M_c}{H_{\text{total}} / M_{\text{total}}}
\end{equation}

where $M_c$ is the number of miners on chain $c$.

\end{document}
